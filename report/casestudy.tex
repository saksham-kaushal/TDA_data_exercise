%Example of use of Time Domain Astrophysics latex class
\documentclass{tda}

%define the page header/title info
\course{Time Domain Astrophysics --- 2020/21 --- Data Exercise} %DON'T EDIT ME!

\begin{document}

\abstract{Here you can provide a brief abstract in support of your Data Exercise. But note, how this document is automatically anonymous!}

\section{Introduction \& Aims}

Provide a brief, but referenced introduction to the Data Exercise. The following sections indicate options you might want to make to help structure your document, these are by no means set in stone!

\section{Methodology}

You might next choose to talk brief about the you methodology. You'll also want to talk about the data too, but you may want to seperate this into...

\subsection{Photometry}

...and... 

\subsection{Spectroscopy}

To make it easier to follow!

\section{Results \& Analysis}

Pretty much as it says on the tin (the section heading), but we might also want to include images, which we reference (see Figure~\ref{fig:example}).

\begin{figure}[h]
    \centering
    \includegraphics{example-grid-100x100pt}
    \caption{Literally nothing but an example image!}
    \label{fig:example}
\end{figure}

Much more information about inserting images and other floats (e.g., tables) can be found online.

Equations can be inserted, e.g.:

\begin{equation}
R_\mathrm{S} = \left( \frac{ 3 S_\star }{ 4 \pi n^2 \beta_2} \right)^\frac{1}{3},
\label{eq:stromgen}
\end{equation}

\noindent and referenced:\ Equation~\ref{eq:stromgen} gives the radius of a Str{\"o}mgen sphere.

\section{Discussion \& Conclusions}

Don't forget to discuss your results in context, and draw critical and quantitative conclusions!

\subsection{Citations}

Of course, throughout, you will have wanted to reference the publications to which you have referred. We can provide citations to support statements \citep{2019Natur.565..460D}, or we can provide citations `in-line':\ e.g., \citet{2019Natur.565..460D} reported something interesting!

Here we have provided one example of how to reference within \LaTeX\ documents, but would recommend referring to on-line resources for more information. To produce references in the format used here, we utilise the NASA ADS website. Each paper abstract on that site has an `Export Citation' option, where we select the reference to be in the `MNRAS' format.

\begin{thebibliography}{99}

\bibitem[\protect\citeauthoryear{Darnley et al.}{2019}]{2019Natur.565..460D} Darnley M.~J., Hounsell R., O'Brien T.~J., Henze M., Rodr{\'\i}guez-Gil P., Shafter A.~W., Shara M.~M., et al., 2019, Natur, 565, 460. doi:10.1038/s41586-018-0825-4

\end{thebibliography}

\end{document}
